\documentclass[a4paper,titlepage]{article}
\usepackage[final]{pdfpages}
\usepackage{hyperref}


\hypersetup{
    bookmarks=true,         % show bookmarks bar?
    colorlinks,%
    citecolor=black,%
    filecolor=black,%
    linkcolor=black,%
    urlcolor=black
}


\title{Project Plan}
\author{Andreas Berggren\\
        Gustav Freij\\
        Christoffer Karlsson}
\date{\today}

\begin{document}

\maketitle

% Table of contents
\pagenumbering{roman}
\tableofcontents
\newpage

% 
\setcounter{page}{1}
\pagenumbering{arabic}

\section{Introduction}
This project plan describes this application and the development of it. It will
cover how our organization is governed and our road to development and completion
of the project.

The project is a virtual take on the classic board game "Labyrinth".
The user will be able to navigate through a labyrinth, made out of walls and
sink holes, using the built-in accelerometer in the android phone.
\section{Organization}
The conditions through which this application is developed are chosen for most
possible throughput from every team member. We will work using a modified version
of the "waterfall" model, making it a hybrid between "scrum" and the "waterfall"
model. The organization consists of three members, making all an equal part
of the project. Together, a basic outline of what the application will behave and
look like is created (the requirements). During development the group will work
as one on the different iterations.
\section{Responsibilities}
In initial work will be put together as a group. When we start coding we will do
this individually. This will work.

The initial work will be put together as a group. This involves the planning of
the project, the requirements, the test plan, a project schedule and all the other
things needed for the project, withstanding the actual coding of the application.
The coding will be divided equally within the organization, with every member
having responisibility of different areas.
\newpage
\section{Milestones}
Our milestones are divided into different versions. Each version implements a new
set of requirements and takes the application one step further towards the first
final release. For information on each requirement, see the requirements document.

\subsection*{0.1 Beta}
Delivered on: Friday, May 4th\\\\ %LV6
Will contain the following requirements:\\
1. System: 1.1, 1.2, 1.3\\
2. Initiation: 2.1, 2.2, 2.4\\
3. Stering: 3.2\\
4. Interaction: 4.2, 4.6\\
5. Activities: 5.1, 5.1.1, 5.1.4

\subsection*{0.2 Beta}
Delivered on: Wednesday, May 9th\\\\ %LV7
Will contain these new requirements:\\
2. Initiation: 2.3, 2.5\\
3. Steering: 3.3\\
4. Interaction: 4.1, 4.5

\subsection*{0.3 Beta}
Delivered on: Sunday, May 13th\\\\ %LV7
Will contain these new requirements:\\
2. Initiation: 2.6, 2.7\\
4. Interaction: 4.3, 4.4\\
6. Menus: 6.1, 6.1.1, 6.1.3, 6.2, 6.2.3, 6.2.4

\subsection*{1.0 Beta}
Delivered on: Wednesday, May 16th\\ %LV8
Will contain these new requirements:\\\\
5. Activities: 5.1.3, 5.1.2, 5.2, 5.2.2, 5.2.3, 5.3, 5.3.1, 5.3.2\\
6. Menus: 6.2.1

\subsection*{1.0 Final}
Delivered on: Friday, May 18th.\\\\ %Final deadline LV8
Will have all bugs that 1.0 Beta contains removed.

\subsection*{1.1}
Delivered if there is time.\\\\
Will contain these new requirements:\\
8. Vibration: 8.1, 8.2, 8.3, 8.4

\subsection*{1.2}
Delivered if there is time.\\\\
Will contain these new requirements:\\
5. Activities: 5.2.1\\
6. Menus: 6.1.2\\
7. Sound: 7.1, 7.2, 7.3, 7.4

\subsection*{2.0}
Delivered if there is time.\\\\
Will contain these new requirements:\\
6. Menus: 6.2.2
% Add more requirements for additional maps

\end{document}
